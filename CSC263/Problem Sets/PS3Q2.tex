% Created 2017-02-27 Mon 20:56
\documentclass[11pt]{article}
\usepackage[utf8]{inputenc}
\usepackage[T1]{fontenc}
\usepackage{fixltx2e}
\usepackage{graphicx}
\usepackage{longtable}
\usepackage{float}
\usepackage{wrapfig}
\usepackage{rotating}
\usepackage[normalem]{ulem}
\usepackage{amsmath}
\usepackage{textcomp}
\usepackage{marvosym}
\usepackage{wasysym}
\usepackage{amssymb}
\usepackage{hyperref}
\tolerance=1000
\author{Anthony Tam}
\date{\today}
\title{PS3Q2}
\hypersetup{
  pdfkeywords={},
  pdfsubject={},
  pdfcreator={Emacs 26.0.50.1 (Org mode 8.2.10)}}
\begin{document}

\maketitle
\tableofcontents

\section{Question Two}
\label{sec-1}
\subsection{Part a}
\label{sec-1-1}
\begin{itemize}
\item The sequence with the largest possible total cost is $n - 1$ insert
statements followed by a single cut statement. This will cause the
worst possible cost to be $2(n - 1)$ or $2n - 2$.
\item This means the upper bound of the cost will be:\\\\
$= \frac{2n - 2}{n}$\\
$= 2 - \frac{2}{n}$
\end{itemize}
\subsection{Part b}
\label{sec-1-2}
\begin{itemize}
\item Each inserted element should be charged \$2.75 This will leave enough
money for an insert to occur at \$0 and collect enough money to allow
for a cut operation. When an element is inserted, it has a carry over
of \$1.75. If n elements are inserted, there is a carry over of \$n$\cdot$1.75.
This added cost allows for n elements to be inserted, and cut can be 
called until k is equal to 0 without running out of money. When calling
cut untill k = 0, there will either be a remainder of 0$\le$X$\le$1.75
\end{itemize}
% Emacs 26.0.50.1 (Org mode 8.2.10)
\end{document}
